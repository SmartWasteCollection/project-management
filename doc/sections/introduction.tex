\section*{Introduzione}
\addcontentsline{toc}{section}{Introduzione}
\label{sec:introduction}

Smart Waste Collection nasce come progetto integrato per i corsi di \textit{Pervasive Computing} e \textit{Laboratorio di Sistemi Software}.
In questa relazione, vengono descritti nel dettaglio i metodi di \textit{project management} utilizzati per lo sviluppo del progetto; in particolare, vengono spiegate nel dettaglio tutte le attività che hanno guidato i processi decisionali e come hanno impattato l'andamento del progetto stesso.
Tra i requisiti del corso di \textit{Laboratorio di Sistemi Software} vi è l'utilizzo di un approccio \textit{Domain Driven}. Questo documento contiene anche gli artefatti prodotti come output di ciascun \textit{process group}, appositamente elencati nell'appendice.

La struttura del documento prevede delle sezioni ($*$) che corrispondono ai \textit{process group} applicati per la gestione del progetto, delle sotto-sezioni ($*.*$) che rappresentano i processi svolti per ogni \textit{process group} ed infine delle sotto-sotto-sezioni ($*.*.*$) che riportano invece le attività operative svolte, nonché gli output scaturiti da tali attività.

Il presente documento è stato sviluppato all'interno di un apposito repository GitHub all'interno di una \href{https://github.com/SmartWasteCollection}{organizzazione} creata per il progetto. Per ulteriori dettagli, consultare l'apposita \href{https://smartwastecollection.github.io/documentation/}{documentazione}.