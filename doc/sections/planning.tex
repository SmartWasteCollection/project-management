\section{Planning}
\label{sec:planning}

In questa sezione viene portata avanti la pianificazione delle fasi successive del progetto. In particolare, verrà definito un modello per il \textit{Project Management Life Cycle} motivandone la scelta e verrà eseguita un'analisi delle attività da svolgere per raggiungere gli obiettivi specificati nella sezione di scoping (\ref{sec:scoping}).

Questa fase è stata condotta in diverse riunioni di \textit{Joint Project Planning Session} con l'intero team di sviluppo.

\subsection{Project Management Life Cycle Model}
In primo luogo, è stato scelto il \textit{life cycle} ritenuto più adatto per condurre il progetto. Si è quindi optato per un modello \textbf{incrementale}. Tale scelta è stata effettuata perché sono noti a priori sia i goal, sia le soluzioni che permettono di raggiungere gli obiettivi; perciò, non è stato ritenuto necessario adottare un approccio agile. Ciò è dovuto anche all'utilizzo di un approccio \textit{domain driven} che ha permesso di studiare a fondo il dominio del problema. Tuttavia, non è stato scelto un approccio lineare (a cascata) poiché la formalizzazione dei requisiti mediante \textit{user stories} rende naturale la scomposizione in \textbf{milestone}.

\subsection{Work Breakdown Structure}
