\section{Planning}
\label{sec:planning}

In questa sezione viene portata avanti la pianificazione delle fasi successive del progetto. In particolare, verrà definito un modello per il \textit{Project Management Life Cycle} motivandone la scelta e verrà definita una schedula contenente le attività da svolgere per raggiungere gli obiettivi specificati nella sezione di scoping (\ref{sec:scoping}).

\subsection{Definizione della Schedula}
Questa fase è stata condotta in diverse riunioni di \textit{Joint Project Planning Session}. In ciascun meeting sono stati presenti tutti i membri core del team, i project manager, un facilitator (che ha anche rappresentato il cliente) e il resource manager.

Nella prima riunione è stato nuovamente presentato lo sponsor del progetto che coincide con il committente, e sono stati presentati tutti i membri core del team, per rafforzare il legame con il cliente. Successivamente, è stato deciso il \textit{Project Management Life Cycle} riflettendo su vantaggi e svantaggi offerti da ciascuna metodologia nel caso del progetto corrente. Questa fase è considerata di fondamentale importanza poiché può influenzare pesantemente le successive riunioni di \textit{planning}.

Nella seconda riunione sono stati validati i requisiti ed è stato individuato un diagramma delle dipendenze tra questi per comprenderne le priorità.

Nei meeting successivi sono state individuate le attività e i task da svolgere per realizzare ogni requisito. A ogni task è stata associata una stima di tempo necessario per la sua implementazione: a partire da queste stime, è stato possibile comprendere a grandi linee la durata di ogni requisito in ore/uomo.

\subsubsection{Project Management Life Cycle Model}
In primo luogo, è stato scelto il \textit{life cycle} ritenuto più adatto per condurre il progetto. Si è quindi optato per un modello \textbf{incrementale}. Tale scelta è stata effettuata perché sono noti a priori sia i goal, sia le soluzioni che permettono di raggiungere gli obiettivi; perciò, non è stato ritenuto necessario adottare un approccio agile. Ciò è dovuto anche all'utilizzo di un approccio \textit{domain driven} che ha permesso di studiare a fondo il dominio del problema. Tuttavia, non è stato scelto un approccio lineare (a cascata) poiché la formalizzazione dei requisiti mediante \textit{user stories} rende naturale la scomposizione in \textbf{milestone}.

\subsubsection{Analisi delle dipendenze}
In seguito alla validazione dei requisiti, sono state individuate le dipendenze realizzative tra di essi. Nello specifico, come mostra il diagramma \refandback{uml/gantt-requirements-dependencies}, il requisito della \textbf{Dumpster Infrastructure} è il primo da portare a termine, seguito dal \textbf{Trucks Routing}. A seguire, possono essere implementati \textbf{Admin Dashboard} e \textbf{Citizen App} parallelamente; tuttavia, si ritiene la \textit{Admin Dashboard} leggermente più prioritaria per riuscire a monitorare al meglio i requisiti precedenti.

\subsubsection{Work Breakdown Structure}
A partire dai requisiti individuati in fase di \textit{scoping}, è stata fatta una scomposizione delle \textbf{attività} da svolgere per portarli a compimento. Da ogni attività sono stati poi dedotti i \textbf{task}, cioè le attività di massimo dettaglio che saranno la base per stimare i tempi e i costi realizzativi di ogni requisito. Ad ogni task (e ad ogni attività che non genera ulteriori task) è stato associato un identificativo univoco per poterci fare riferimento in momenti successivi.
Si riportano quindi le \textit{Work Breakdown Structure} di:
\begin{enumerate}
    \item \textbf{Dumpster Infrastructure} (\refandback{uml/wbs-dumpster-infrastructure}).
    \item \textbf{Trucks Routing} (\refandback{uml/wbs-trucks-routing}).
    \item \textbf{Admin Dashboard} (\refandback{uml/wbs-admin-dashboard}).
    \item \textbf{Citizen App} (\refandback{uml/wbs-citizen-app}).
\end{enumerate}

\subsubsection{Diagrammi Gantt}
\'E stata effettuata un'analisi che ha permesso di produrre dei diagrammi Gantt per ogni requisito individuato. In particolare, è stata associata una stima di tempo necessario per svolgere ogni task; successivamente, sono state identificate le dipendenze tra di essi: queste permettono di comprendere quali task possono essere svolti in parallelo e quali invece necessitano di averne portati a termine altri prima di poter essere cominciati. Si riportano quindi i diagrammi di Gantt realizzati:
\begin{itemize}
    \item \textbf{Dumpster Infrastructure} (\refandback{uml/gantt-dumpster-infrastructure}).
    \item \textbf{Trucks Routing} (\refandback{uml/gantt-trucks-routing}).
    \item \textbf{Admin Dashboard} (\refandback{uml/gantt-admin-dashboard}).
    \item \textbf{Citizen App} (\refandback{uml/gantt-citizen-app}).
\end{itemize}
Ogni diagramma è suddiviso in sezioni parallele che rappresentano le funzioni individuate nella \textit{Requirement Breakdown Structure} (\ref{sec:rbs}). Inoltre, la durata concordata per ogni task è rappresentata in ore/uomo, considerando però che ogni unità lavorativa svilupperà in \textit{pair programming}: in passato, è stato dimostrato che le risorse a disposizione hanno una produttività maggiore quando in coppia; di conseguenza, i project manager hanno proposto tale modalità operativa già in fase di \textit{planning} e il committente ha concordato la considerazione nella pianificazione delle attività.
