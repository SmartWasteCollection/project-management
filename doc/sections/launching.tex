\section{Launching \& Executing}
\label{sec:launching}

In questa sezione vengono riportati gli elementi significativi del \textit{launch} del progetto.

\subsection{Definizione del Team e delle Regole Operative}

Il team di progetto è stato idealmente definito già in fase di \textit{scoping}, di conseguenza il processo di recruiting non ha richiesto molto tempo. Quindi, una volta confermato il team che ha partecipato alle fasi precedenti, sono stati definiti gli aspetti operativi che accompagnano le parti successive.

In seguito, è stato tenuto un \textit{Kick-Off Meeting} che ha avuto l'obiettivo di riunire tutti gli \textit{stakeholder} del progetto per ripresentare gli aspetti rilevanti del lavoro da svolgere, sotto la prospettiva della pianificazione schedulata.

\subsubsection{Staffing}
I seguenti membri del team, hanno partecipato a tutte le riunioni precedenti e sono, di conseguenza, allineati sulla pianificazione del progetto:
\begin{itemize}
    \item Martina Baiardi (\textit{Converging})
    \item Alessandro Marcantoni (\textit{Accomodating})
    \item Simone Romagnoli (\textit{Diverging})
    \item Marta Spadoni (\textit{Assimilating})
\end{itemize}
Vista l'approvazione della pianificazione da parte del cliente, non è stato ritenuto necessario espandere ulteriormente il \textit{core team}, che, pertanto, coincide con il \textit{developer team}.

\subsubsection{Resource Allocation}
In primo luogo, \textit{Alessandro} e \textit{Simone} hanno svolto da \textit{project manager}, quindi sono stati incaricati di tutti gli aspetti non meramente implementativi.
Per quanto riguarda invece gli aspetti più implementativi, \textit{Martina} è stata eletta responsabile delle scelte tecnologiche, che verranno tuttavia confermate dai project manager e discusse assieme al developer team durante le varie riunioni. Invece, \textit{Marta} è stata eletta responsabile del prodotto: si è quindi occupata di gestire i repository di codice prodotti e ha definito come assicurarsi un'alta qualità del codice attraverso processi di \textit{continuous integration} e \textit{continuous delivery}.

\subsubsection{Operative Rules}
Sono state stabilite delle regole operative per allineare i vari membri del team sulle modalità d'esecuzione dei processi.
\begin{itemize}
    \item \textbf{Problem Solving} - Per risolvere eventuali problemi ci si affiderà a un approccio \textit{team-oriented}. Innanzitutto, verranno discussi durante i meeting giornalieri e si cercherà insieme di individuare le cause e i "proprietari" dei problemi. Una volta analizzato al meglio il problema, verranno generate idee risolutive con dei \textit{brainstorming} e si svilupperanno dei piani d'azione sulla base di quelle migliori.
    \item \textbf{Decision Making} - Il processo di \textit{decision making} avrà un approccio di tipo \textbf{consultativo}: i project manager saranno ritenuti responsabili delle decisioni prese e delle eventuali conseguenze, tuttavia queste verranno prese solamente dopo essere state discusse con il team.
    \item \textbf{Meetings} - Ogni mattina verranno condotte degli \textbf{stand-up meeting} che avranno lo scopo di condividere le realizzazioni di task e il sorgimento di eventuali problemi. Inoltre, verranno condotte delle riunioni prima di iniziare la realizzazione di un nuovo requisito e alla fine di questa per valutare il lavoro svolto. Ovviamente, sarà possibile richiedere dei meeting per qualsiasi membro e in qualsiasi momento, purché sia presentata una valida motivazione.
    \item \textbf{Comunicazioni} - Sono stati concordati diversi canali di comunicazione sulla base degli argomenti di discussione:
    \begin{itemize}
        \item \textit{Whatsapp} per le comunicazioni più effimere.
        \item \textit{Microsoft Teams} per le comunicazioni d'ufficio
        \item \textit{Email} per le comunicazioni più formali.
        \item \textit{Slack} per le comunicazioni più tecniche.
    \end{itemize}
    Il cliente ha espressamente dichiarato la sua preferenza per le chiamate (telefoniche o in videoconferenza) in caso di comunicazioni che lo coinvolgono. Tuttavia, è stato comunque integrato nei canali più interni affinchè sia reso partecipe anche delle notizie minori relative al progetto.
\end{itemize}

\subsubsection{Project Impact Statement}
