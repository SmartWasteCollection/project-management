\section{Monitoring \& Control}
\label{sec:monitoring}

In questa sezione vengono riportate le strategie adottate per monitorare i task definiti in fase di pianificazione (\ref{sec:planning}). Inoltre, viene riportato il processo di accettazione del prodotto da parte del committente.

\subsection{Monitoraggio del Progetto}
Il progetto è stato monitorato da diversi punti di vista. Tale attività è stata ritenuta necessaria per rispettare la pianificazione effettuata ed accorgersi presto di eventuali problematiche. Nello specifico, sono stati utilizzati dei metodi formali per monitorare sia lo stato di avanzamento dei task, sia lo stato dei membri del team in termini di problematiche riscontrate e di produttività.

\subsubsection{Progress Reporting}
Per riportare lo stato di avanzamento dei task, il team ha deciso di affidarsi a due sistemi di reporting; in particolare:
\begin{itemize}
    \item \textbf{Trello}, per riportare lo svolgimento delle attività più recenti (\textit{current period reports});
    \item \textbf{Boardify}, per riportare l'avanzamento del progetto coprendone l'intera storia (\textit{cumulative reports}).
\end{itemize}
Entrambi i sistemi scelti permettono di istanziare delle bacheche sulle quali è possibile tracciare l'andamento dei task e individuare lo scostamento tra lo stato di avanzamento attuale e la pianificazione. Inoltre, il secondo sistema è stato sviluppato internamente all'azienda: di conseguenza, i membri del developer team si sono trovati già familiari con tale strumento e ciò ha permesso di ridurre l'\textit{overhead} introdotto dal processo di monitoring.

In aggiunta, il responsabile del prodotto (\textit{Marta}) ha considerato il processo di monitoring come fondamentale per lo sviluppo di codice. Quindi, è stato concordato di inserire tale processo nella \textbf{continuous integration} del prodotto. Infatti, è stato possibile stilare dei report relativi a ciascun task grazie ai messaggi di \textit{commit} effettuati dagli sviluppatori; in questo modo, è stato possibile automatizzare il controllo dello stato di avanzamento in ogni task senza introdurre maggiore \textit{overhead}.

\subsubsection{Issues Log}
Per tenere traccia di tutte le problematiche che emergono è stato redatto un documento chiamato \textit{Issues Log}. In particolare, tale documento ha permesso di identificare e analizzare meglio le problematiche, ed è stato utilizzato per evidenziare, in una determinata fase del progetto, quali problemi si trovassero in risoluzione e quali fossero già risolti. Inoltre, la condivisione di tale documento con il committente ha permesso di coinvolgerlo maggiormente e di incrementare la sua fiducia nei confronti del team.

Dato un problema, l'\textit{Issues Log} ha permesso di tracciare:
\begin{itemize}
    \item nome del problema;
    \item descrizione del problema;
    \item proprietario del problema;
    \item data di sorgimento;
    \item data di risoluzione;
    \item metodi risolutivi ipotetici;
    \item metodo risolutivo utilizzato;
    \item stato attuale.
\end{itemize}

\subsubsection{Meetings}
Come anticipato nella sezione \ref{sec:planning-operative-rules}, quotidianamente sono stati condotti dei \textbf{Project Status Meeting}, ai quali ha partecipato ogni membro del developer team. In queste riunioni non è stato incluso il committente, tuttavia è stato sempre riportato l'esito con una breve descrizione del meeting (massimo 10 righe). Il programma di queste riunioni prevede:
\begin{itemize}
    \item breve presentazione dello stato di avanzamento dei task svolti dai membri del developer team;
    \item aggiornamento dell'\textit{Issues Log};
    \item aggiornamento (se necessario) della schedula;
    \item aggiornamento del \textbf{Project Notebook}.
\end{itemize}

\subsection{Accettazione del Progetto}
Una volta terminata la realizzazione di tutti i requisiti, il team si è assicurato che il sistema funzionasse nella sua interezza: ciò è stato possibile anche durante lo svolgimento del progetto grazie ai test sviluppati e ai processi di \textit{continuous integration} e \textit{continuous delivery}; tuttavia, si è voluto ottenere un'ulteriore conferma a progetto terminato riprovando i test e verificando il giusto funzionamento del deployment.

Successivamente, è stato condotto un meeting, come da pianificazione, per mostrare al committente il progetto terminato. Costui ha espresso la sua soddisfazione e ha concordato a procedere con la fase di chiusura del progetto.