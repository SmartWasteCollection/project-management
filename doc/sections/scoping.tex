\section{Scoping}
\label{sec:scoping}

Per svolgere la fase di scoping al meglio, si simula che il progetto Smart Waste Collection venga proposto dal cliente \textit{Sphera}.
In questo scenario, noi svolgiamo il ruolo di software house e, nello specifico, ricopriamo la posizione di \textit{project manager} con diverse esperienze pregresse.

Nei seguenti capitoli, verrà descritta la richiesta del progetto da parte del cliente, per poi effettuarne un'analisi rappresentata dal Project Overview Statement; tale documento sarà d'interesse sia per i \textit{senior manager} della nostra azienda, sia per quelli dell'azienda cliente.

\subsection{Proposta del Cliente}
\textit{Sphera} è un'azienda multiutility leader nei servizi ambientali, idrici ed energetici nell'ambito di una regione italiana.
Negli ultimi due anni ha riscontrato lamentele da parte dei clienti in merito al servizio di smaltimento dei rifiuti.
In particolare, i cittadini hanno manifestato malcontento a causa dell'inefficienza del servizio stesso e di periodi di
mancato servizio, in cui non sono stati in grado di conferire i rifiuti.
Questo problema è dato dal fatto che i cassonetti sono talvolta pieni, tuttavia la raccolta programmata è prevista dopo alcuni giorni.
\textit{Sphera} stessa si è inoltre accorta che alcune delle raccolte programmate hanno prelevato un quantitativo di rifiuti non sufficiente a giustificare la mobilitazione di un camion.
Inoltre, un altro problema evidenziato dai cittadini consiste nello smaltimento di rifiuti straordinari (ferro, sterpaglie, ecc.).
Nello specifico, al momento i cittadini, per smaltire i rifiuti straordinari, si devono personalmente recare presso il punto di smaltimento a loro più vicino, previa prenotazione telefonica al numero verde di \textit{Sphera}.
Le lamentele riguardano principalmente la disponibilità del servizio di call-center e i lunghi tempi d'attesa per l'effettivo smaltimento.
Si richiede quindi un sistema che permetta il monitoraggio dei bidoni stessi indicando così all'azienda il momento più opportuno per la raccolta.
In particolare, il manager dei servizi ambientali Richie Clato ha proposto di introdurre dei nuovi cassonetti con sensori che permettono di misurare il volume e il peso occupati dai rifiuti.
Inoltre, tali cassonetti possono essere aperti solamente tramite l'utilizzo di una smart card fornita ai cittadini.
Date queste premesse, Richie ha identificato i seguenti sistemi software che compongono la soluzione:
\begin{itemize}
    \item \textbf{Dumpster Infrastructure}: i cassonetti saranno dotati di sensori che ne monitorano vari parametri.
    È pertanto necessaria un'infrastruttura che consenta di visualizzare in tempo reale lo stato dei cassonetti sparsi per il territorio.
    La soluzione deve essere scalabile in quanto deve astrarre dal numero di cassonetti.
    Deve inoltre mantenere lo storico degli accessi dei cittadini per futuri utilizzi, a partire da indagini statistiche fino a includere premi per i cittadini modello.
    \item \textbf{Citizen App}: un'applicazione mobile che permetta al cittadino di consultare lo stato dei cassonetti in ciascuna stazione di raccolta per sapere se è possibile conferire un particolare tipo di rifiuto.
    L'applicazione deve presentare una vista "a mappa della città" per permettere al cittadino di orientarsi.
    Inoltre, sarebbe opportuno se nella \textit{home} il cittadino potesse vedere lo stato di alcune stazioni preferite.
    \item \textbf{Trucks Routing}: un'applicazione connessa con la \textbf{Dumpster Infrastructure} che organizza gli itinerari dei camioncini sulla base dello stato dei cassonetti in una determinata zona.
    Quando un autista parte dalla filiera deve avere l'itinerario pronto e deve raccogliere solamente un tipo di rifiuto.
    \item \textbf{Extraordinary Waste Booking}: un'applicazione o sito che permetta ai cittadini di prenotare lo smaltimento di rifiuti straordinari a casa: il cittadino deve indicare l'indirizzo di casa propria, il tipo di rifiuto straordinario e i giorni di disponibilità per la raccolta.
    \item \textbf{Admin Dashboard}: un'applicazione che permetta di visualizzare lo stato di cassonetti e posizione dei camioncini in tempo reale per gli amministratori.
    Deve inoltre poter visualizzare i dati storici e statistiche relative a conferimenti effettuati da cittadini e prenotazioni per raccolte di rifiuti straordinari.
\end{itemize}

\subsection{Analisi del Dominio}
Sono state condotti 3 \textit{Project Scoping Meetings} con lo scopo di studiare al meglio il dominio in questione, in particolare adottando un approccio \textit{Domain Driven}.

Nella prima riunione è stata discussa la proposta del cliente che, pertanto, è stato coinvolto in prima persona.
Al fine di chiarire eventuali dubbi ed ambiguità presenti nella proposta stessa, è stata inoltre condotta un'intervista (\ref{sec:interview}): a questa hanno partecipato, oltre ovviamente al cliente, i project manager, i membri core del team ed un facilitator. Quest'ultima figura è stata messa a disposizione direttamente dal cliente poiché presenta competenze sia tecniche che riguardanti il dominio del problema.

In seguito, è stata condotta una seconda riunione con lo scopo di analizzare in maniera approfondita il dominio.
A questa riunione hanno partecipato i project manager, i membri core del team e il facilitator del cliente; l'obiettivo di questo meeting era quello di tradurre i desideri del cliente in user stories e requisiti formali.
L'output di questa riunione ha compreso, oltre a user stories e \textit{Requirements Breakdown Structure}, anche documenti formali utili successivamente al team tra cui \textit{Impact Mapping}, \textit{Benefit Mapping}, diagrammi dei casi d'uso e un vocabolario comune sotto forma di \textit{Ubiquitous Language}. Inoltre, è stato possibile decidere un \textit{Project Management Life Cycle Model} a partire dall'incertezza dei requisiti.

L'obiettivo della terza ed ultima riunione è stato quello di stilare un documento di \textit{Project Overview Statement} in modo da formalizzare definitivamente il progetto.
A questa hanno partecipato, proprio come nella prima, il cliente, i project manager, i membri core del team e il facilitator.

\subsubsection{Intervista col Committente}\label{sec:interview}
Si riporta di seguito l'intervista con il committente.

\question{1}{Com'è attualmente gestita la raccolta dei rifiuti?}
\answer{Al momento i camioncini raccolgono periodicamente i rifiuti dalle stazioni di raccolta. Noi abbiamo una schedula che elenca quali tipi di rifiuto dobbiamo raccogliere in un dato giorno. Ogni camioncino parte dalla filiera di smaltimento e svuota i cassonetti di tutti i punti di raccolta nella sua \textit{mission area}.
Noi vorremmo ottimizzare la raccolta per permettere ai cittadini di conferire i rifiuti in qualsiasi momento presso il punto di raccolta più vicino.}

\question{2}{Quanti cassonetti ci sono in ogni punto di raccolta?}
\answer{In base al numero di persone che vivono nella zona del punto di raccolta, possono esserci uno o più cassonetti per ogni tipo di rifiuto comune. Per esempio, un punto di raccolta può essere composto da due cassonetti della plastica, due della carta e uno per ogni altro tipo.}

\question{3}{Quali tipi di cassonetti ci sono in ogni punto di raccolta?}
\answer{I tipi di rifiuto presenti in ogni punto di raccolta sono indifferenziata, organico, plastica/alluminio, carta e vetro; tuttavia, stiamo pensando di introdurre altri tipi di rifiuto.
I cassonetti possono avere caratteristiche fisiche diverse, a seconda dal tipo di rifiuto raccolto. Infatti, ci sono due grandezze possibili per i cassonetti: una grande, adatta per rifiuti che non necessitano di essere raccolti con alta frequenza, e una più piccola, che viene usata per i rifiuti organici, ad esempio.
Al momento, i cassonetti più grandi possono essere aperti utilizzando una leva ad altezza dei piedi, mentre quelli più piccoli si aprono manualmente. Tuttavia, vorremmo che i cittadini fossero in grado di aprire tutti i cassonetti utilizzando una smart card. }

\question{4}{Quali tipi di rifiuto vorreste raccogliere? Come?}
\answer{Vorremmo prelevare i rifiuti comuni direttamente dai punti di raccolta. D'altro canto, vorremmo fornire un servizio di raccolta "a casa" \textit{on-demand} per i rifiuti straordinari.
I rifiuti comuni includono:
\begin{itemize}
    \item carta
    \item plastica/alluminio
    \item vetro
    \item indifferenziata
    \item organico
\end{itemize}
I rifiuti straordinari includono:
\begin{itemize}
    \item ramaglie
    \item olio esausto
    \item ferro
    \item elettronica
    \item vestiti
    \item altro
\end{itemize}
}

\question{5}{Quali sono le principali caratteristiche dei cassonetti?}
\answer{Le principali caratteristiche sono:
\begin{itemize}
    \item grandezza: in termini del volume di rifiuti che può contenere (misura in litri)
    \item colore: indica il tipo di rifiuto raccolto
    \item apertura: può essere tramite una leva ad altezza piedi o manuale
\end{itemize}
}

\question{6}{Quali informazioni vorreste tracciare per quanto riguarda un singolo cassonetto?}
\answer{Per ogni cassonetto vorremmo sapere:
\begin{itemize}
    \item il volume occupato
    \item se è aperto o chiuso
    \item se è funzionante o danneggiato
    \item se deve essere svuotato
\end{itemize}
}

\question{7}{Quali tipi di camioncini dei rifiuti avete? Quanti tipi di rifiuti può prelevare un singolo camioncino?}
\answer{Abbiamo solo un tipo di camioncini. Durante una missione, può raccogliere solo un singolo tipo di rifiuto.
Stiamo anche pianificando di acquisire camioncini specifici per i rifiuti straordinari per soddisfare le richieste dei clienti.}

\question{8}{Quanti camioncini dei rifiuti avete?}
\answer{Ce ne sono circa da 10 a 30 parcheggiati in ogni filiera di smaltimento. L'ammontare preciso dipende dal numero di punti di raccolta nella provincia della filiera.}

\question{9}{Quali informazioni vorreste tracciare per quanto riguarda un singolo camioncino dei rifiuti?}
\answer{Vorremmo tracciare:
\begin{itemize}
    \item il volume totale
    \item il volume attualmente occupato
    \item il tipo di rifiuto che si sta raccogliendo al momento
    \item la sua posizione in tempo reale
\end{itemize}
}

\question{10}{Dove vengono portati i rifiuti dai camioncini?}
\answer{I rifiuti vengono portati nella filiera di smaltimento della provincia.}

\question{11}{Quindi i camioncini raccolgono rifiuti da ogni punto di raccolta della provincia?}
\answer{No. A ogni camioncino viene assegnata una missione e, di conseguenza, una \textit{mission area}. Una \textit{mission area} è composta da un insieme di aree residenziale che sono fisicamente vicine tra loro.}

\question{12}{Come sono distribuiti i punt di raccolta nel territorio?}
\answer{C'è una filiera di smaltimento per ogni provincia. Ogni filiera è responsabile dei rifiuti prelevati dai punti di raccolta della propria provincia. Una provincia è suddivisa in aree residenziali dove è collocato un singolo punto di raccolta. Le aree residenziali sono dimensionate in modo da "coprire" più o meno lo stesso numero di persone.}

\question{13}{Quali tipi di rifiuto vengono gestiti in una filiera di smaltimento?}
\answer{Ogni filiera gestisce tutti i tipi di rifiuti. Ciascun tipo di rifiuto viene smaltito da un'apposita catena di smaltimento all'interno della filiera.}

\question{14}{Come volete gestire i rifiuti straordinari?}
\answer{Al momento, i rifiuti straordinari vengono portati direttamente alle filiere dai cittadini. Tuttavia, vorremmo che i cittadini siano in grado di prenotare un appuntamento per raccogliere i rifiuti straordinari direttamente a casa loro.}

\subsubsection{Ubiquitous Language}
Per evitare ogni genere di ambiguità all'interno del dominio studiato, è stato accordato un vocabolario comune, identificato come \textit{Ubiquitous Language}. In particolare, sono stati individuati i termini più usati dal cliente e spiegati con la definizione del cliente quando fornita. I termini sono stati raggruppati per \textit{topic}, nello specifico:
\begin{itemize}
    \item \textit{Ubiquitous Language} del \textbf{cittadino} (\refandback{citizen-ubiquitous-language}).
    \item \textit{Ubiquitous Language} della \textbf{raccolta} (\refandback{collection-ubiquitous-language}).
    \item \textit{Ubiquitous Language} del \textbf{cassonetto} (\refandback{dumpster-ubiquitous-language}).
    \item \textit{Ubiquitous Language} del \textbf{camioncino} (\refandback{truck-ubiquitous-language}).
    \item \textit{Ubiquitous Language} dei \textbf{rifiuti} (\refandback{waste-ubiquitous-language}).
\end{itemize}

\subsubsection{User Stories}
Dall'intervista col cliente sono scaturiti quelli che saranno i 3 attori principali nel sistema da realizzare: i manager dell'azienda cliente, i cittadini e gli autisti dei camion dei rifiuti. Le \textit{user stories} prodotte (elencate di seguito) sono state quindi realizzate concentrandosi su questi attori.

\storyactor{Manager}
\begin{story}
    ... \textit{I want to} observe the real-time position of garbage trucks and the type of waste they are carrying \textit{so that} I can monitor active missions.
\end{story}
\begin{story}
    ... \textit{I want to} observe the list of complaints received from citizens and dumpsters \textit{so that} I can fix possible issues.
\end{story}
\begin{story}
    ... \textit{I want to} observe collection points and dumpsters' status \textit{so that} I can check whether the system is working or not.
\end{story}
\begin{story}
    ... \textit{I want to} observe disposal points' position \textit{so that} I can have a visual representation of their location.
\end{story}
\begin{story}
    ... \textit{I want to} observe the list of "at home" collection requests \textit{so that} I can verify the usefulness of the service.
\end{story}
\begin{story}
    ... \textit{I want to} create a new smart card for specific citizens \textit{so that} they can open dumpsters with it.
\end{story}

\storyactor{Citizen}
\begin{story}
    ... \textit{I want to} open dumpsters with my smart card \textit{so that} I can dispose waste effortlessly and without touching the dumpster.
\end{story}
\begin{story}
    ... \textit{I want to} book an "at home" waste collection \textit{so that} I don't have to go to the disposal point.
\end{story}
\begin{story}
    ... \textit{I want to} observe all collection points \textit{so that} I can see the types of waste that I can dispose of, the percentage of occupied volume for every dumpster and whether they are available or not.
\end{story}
\begin{story}
    ... \textit{I want to} report issues \textit{so that} I can help improve the service.
\end{story}

\storyactor{Truck Driver}
\begin{story}
    ... \textit{I want to} automatically receive missions \textit{so that} I can know which disposal points and which type of waste to collect.
\end{story}
\begin{story}
    ... \textit{I want to} automatically receive "at home" missions \textit{so that} I can know which extraordinary waste collection requests I have to satisfy.
\end{story}
\begin{story}
    ... \textit{I want to} report issues \textit{so that} I can notify managers about possible problems.
\end{story}

\subsubsection{Impact Mapping}
Il diagramma di \textit{impact mapping} realizzato (\refandback{impact-mapping}) aiuta ad individuare una prima versione del \textbf{goal}. Inoltre, permette di specificare quali sono gli attori del dominio studiato e, per ognuno di questi, quali sono le variazioni di maggiore impatto introdotte dal progetto. In questo modo si riescono a dedurre delle idee per eventuali \textit{deliverable} da proporre nel \textit{Project Overview Statement}. Infine, il diagramma di \textit{impact mapping} può fungere da spunto per la \textit{Requirement Breakdown Structure}.

Inoltre, è stato prodotto un diagramma di \textit{benefit mapping} (\refandback{benefit-mapping}) realizzato a partire dalle \textit{user stories}. Questo mostra chi sono gli attori che beneficiano maggiormente dai cambiamenti introdotti dal sistema.

\subsubsection{Casi d'Uso}
Sulla base delle user stories individuate, sono stati prodotti dei diagrammi dei casi d'uso per meglio formalizzare i requisiti. In particolare, sono stati analizzati gli scenari:
\begin{itemize}
    \item \textbf{Ordinary waste disposal} (\refandback{ordinary-disposal-use-cases}): lo scenario in cui il cittadino si reca presso il \textit{collection point} per conferire i propri rifiuti; gli attori coinvolti in tale caso d'uso sono il cittadino e il cassonetto dei rifiuti.
    \item \textbf{Ordinary waste collection} (\refandback{ordinary-collection-use-cases}): il processo che prevede la missione di raccolta dei rifiuti all'interno dei cassonetti da parte dei camioncini.
    \item \textbf{Extraordinary waste management} (\refandback{extraordinary-management-use-cases}): la gestione dei rifiuti straordinari mediante raccolta a casa dei cittadini da parte dei camioncini.
    \item \textbf{Dashboard} (\refandback{dashboard-use-cases}): l'utilizzo della dashboard da parte dei manager e della \textbf{Citizen App} da parte dei cittadini.
    \item \textbf{Complaints} (\refandback{complaints-use-cases}): la presentazione di reclami da parte di cittadini, autisti dei camioncini e cassonetti e successiva visualizzazione da parte dei manager.
\end{itemize}

\subsubsection{Requirement Breakdown Structure}\label{sec:rbs}
Una volta analizzati i casi d'uso e le \textit{user stories}, è stato prodotto un diagramma di \textit{Requirement Breakdown Structure} \refandback{requirement-breakdown-structure}. Questo permette di descrivere ogni requisito in termini delle funzionalità che questo introduce.

Il \textbf{goal} che accomuna i requisti è l'ottimizzazione della raccolta dei rifiuti. I requisiti che ne scaturiscono coincidono con gli elementi di primo livello; vi sono poi le funzioni di ciascuno, che spiegano quali sono le funzionalità di più alto livello di ciascun requisito. Le funzioni si suddividono ulteriormente in delle feature più specifiche.

\begin{itemize}
    \item \textbf{Ottimizzazione della raccolta dei rifiuti}
    \begin{enumerate}[label*=\arabic*.]
        \item \textbf{Dumpster Infrastructure}
        \begin{enumerate}[label*=\arabic*.]
            \item Apertura dei cassonetti con smart-cards
            \begin{enumerate}[label*=\arabic*.]
                \item Autenticazione delle smart-card
                \item Apertura e chiusura di cassonetti
            \end{enumerate}
            \item Raccolta e condivisione dati
            \begin{enumerate}[label*=\arabic*.]
                \item Aggiornamento del volume occupato
                \item Salvataggio di informazioni sul conferimento dei cittadini
                \item Segnalazione problemi
            \end{enumerate}
        \end{enumerate}
        \item \textbf{Citizen App}
        \begin{enumerate}[label*=\arabic*.]
            \item Verifica dello stato dei cassonetti
            \begin{enumerate}[label*=\arabic*.]
                \item Interfaccia grafica con cassonetti
                \item Verifica della disponibilità dei cassonetti
                \item Verifica del volume occupato dei cassonetti
            \end{enumerate}
            \item Prenotazione di smaltimento di rifiuti straordinari
            \begin{enumerate}[label*=\arabic*.]
                \item Creazione di una richiesta di smaltimento di rifiuti straordinari
            \end{enumerate}
            \item Segnalazione problemi
            \begin{enumerate}[label*=\arabic*.]
                \item Scrivere una lamentela
                \item Inviare lamentele
            \end{enumerate}
        \end{enumerate}
        \item \textbf{Trucks Routing}
        \begin{enumerate}[label*=\arabic*.]
            \item Pianificare missioni di smaltimento di rifiuti ordinari
            \begin{enumerate}[label*=\arabic*.]
                \item Inviare notifica di "cassonetto pieno"
                \item Ricerca di altri cassonetti "quasi pieni" nei dintorni
                \item Calcolo del percorso migliore
                \item Tracciamento della posizione in tempo reale dei camioncini
            \end{enumerate}
            \item Pianificare missioni di smaltimento di rifiuti straordinari
            \begin{enumerate}[label*=\arabic*.]
                \item Notifica di richiesta di smaltimento di rifiuti straordinari
                \item Raggruppamento di missioni per tipo di rifiuto straordinario
                \item Calcolo del percorso migliore
                \item Tracciamento della posizione in tempo reale dei camioncini
            \end{enumerate}
        \end{enumerate}
        \item \textbf{Admin Dashboard}
        \begin{enumerate}[label*=\arabic*.]
            \item Mostrare lo stato dei punti di raccolta
            \begin{enumerate}[label*=\arabic*.]
                \item Mostrare la disponibilità dei cassonetti
                \item Mostrare il volume occupato dei cassonetti
            \end{enumerate}
            \item Mostrare lo stato in tempo reale dei camioncini
            \begin{enumerate}[label*=\arabic*.]
                \item Mostrare la posizione dei camioncini
                \item Mostrare il volume occupato dei camioncini
            \end{enumerate}
            \item Mostrare la missioni
            \begin{enumerate}[label*=\arabic*.]
                \item Mostrare la lista di missioni attive
            \end{enumerate}
            \item Mostrare le segnalazioni
            \begin{enumerate}[label*=\arabic*.]
                \item Mostrare la lista di segnalazioni
            \end{enumerate}
            \item Registrazione di nuovi cittadini
            \begin{enumerate}[label*=\arabic*.]
                \item Creazione di nuove smart-card per i cittadini
            \end{enumerate}
        \end{enumerate}
    \end{enumerate}
\end{itemize}

\subsubsection{Project Management Life Cycle Model}
In seguito all'individuazione dei requisiti, è stato scelto il \textit{life cycle model} ritenuto più adatto per condurre il progetto. Dato il livello di conoscenza dei requisiti in questa fase, si è optato per un modello \textbf{tradizionale incrementale}.
Tale scelta è stata effettuata perché sono noti a priori sia i goal, sia le soluzioni che permettono di raggiungere gli obiettivi; perciò, non è stato ritenuto necessario adottare un approccio agile.
Ciò è dovuto anche all'utilizzo di un approccio \textit{domain driven} che ha permesso di studiare a fondo il dominio del problema.
Tuttavia, non è stato scelto un approccio lineare (a cascata) poiché la formalizzazione dei requisiti mediante \textit{user stories} rende naturale la scomposizione in \textbf{milestone}.

\subsection{Project Overview Statement}
In seguito all'analisi del dominio, è stato prodotto un \textit{Project Overview Statement} per ricevere l'approvazione da parte del \textit{senior management} a procedere con il progetto. Si assume che il \textit{POS} sia stato ben compreso dai \textit{senior manager} e che questi abbiano subito approvato l'inizio del progetto. Tale documento è riportato di seguito:

\newpage
\begin{tabular}{| p{.25\textwidth} | p{.3\textwidth} | p{.30\textwidth} | }
    \hline
    \begin{tabular}{p{.25\textwidth}}\textbf{PROJECT OVERVIEW STATEMENT}\\\end{tabular} &
    \begin{tabular}{p{.3\textwidth}}{\scriptsize Nome Progetto} \\Smart Waste Collection\end{tabular} &
    \begin{tabular}{p{.30\textwidth}}{\scriptsize Project Managers} \\Alessandro Marcantoni,\\Simone Romagnoli\end{tabular} \\
    \hline
\end{tabular}

\possec{Problemi/Opportunità}{
    Il servizio di smaltimento dei rifiuti dell'azienda \textit{Sphera} è inefficiente: talvolta i cassonetti sono pieni, pertanto i cittadini non riescono a conferire i propri rifiuti, e la raccolta programmata avverrà solo alcuni giorni dopo.\\
    Inoltre, la clientela è fortemente insoddisfatta del metodo di raccolta dei rifiuti straordinari che al momento prevede una prenotazione al numero verde per potersi successivamente recare di persona al punto di smaltimento.\\
    L'azienda potrebbe infine sfruttare l'ottimizzazione della raccolta dei rifiuti per minimizzare le missioni di raccolta che prelevano un quantitativo di rifiuti non abbastanza elevato: \textit{Sphera} stessa ha infatti rilevato che alcune raccolte programmate hanno prelevato un quantitativo di rifiuti non sufficiente a giustificare la missione di raccolta.
}

\possec{Goal}{
    L'obiettivo principale è ottimizzare la raccolta rifiuti facendo in modo che, quando almeno un cassonetto raggiunge il 75\% di volume occupato, parta una missione di raccolta che svuoti anche i cassonetti pieni più del 50\% in aree limitrofe.\\
    Parallelamente, si vuole riorganizzare la raccolta di rifiuti straordinari in modo che il 95\% degli appuntamenti venga concordato per via telematica e la raccolta avvenga a casa del cliente.
}

\possec{Obiettivi}{
    Si vogliono realizzare i seguenti sottosistemi:
    \begin{itemize}
        \item \textbf{Dumpster Infrastructure}: dotare i cassonetti di sensori e attuatori in grado di:
        \begin{itemize}
            \item monitorare il volume occupato;
            \item aprire il cassonetto all'avvicinamento di una smart card;
            \item memorizzare/condividere i dati raccolti.
        \end{itemize}
        \item \textbf{Citizen App}: applicazione mobile che permetta al cittadino di:
        \begin{itemize}
            \item consultare lo stato di riempimento dei cassonetti nei punti di raccolta;
            \item prenotare un appuntamento per smaltimento di rifiuti straordinari;
            \item segnalare eventuali problemi/reclami.
        \end{itemize}
        \item \textbf{Trucks Routing}: sistema reattivo che organizza missioni di raccolta rifiuti (ordinari e straordinari) .
        \item \textbf{Admin Dashboard}: piattaforma che consente ai manager di visualizzare informazioni relative ai cassonetti, posizione dei camioncini in tempo reale e una lista dei reclami.
    \end{itemize}
}

\possec{Criteri di Successo}{
    \begin{enumerate}
        \item I cassonetti dei rifiuti non raggiungono mai il 100\% di volume occupato (\textit{Improved Service}).
        \item Il numero medio di missioni mensili si riduce di almeno del 25\% (\textit{Avoided Cost}).
        \item Non si ricevono più di 5 reclami al mese (\textit{Improved Service}).
        \item Lo smaltimento di rifiuti straordinari avviene, almeno al 95\%, in seguito a prenotazioni sulla \textbf{Citizen App} (\textit{Improved Service}).
    \end{enumerate}
}

\possec{Rischi/Assunzioni/Ostacoli}{
    \begin{itemize}
        \item Si assume che l'azienda \textit{Sphera} possa sostenere i costi dovuti alla realizzazione dell'intero progetto.
        \item Si assume che i rifiuti conferiti da un singolo cittadino non superino il 10\% della capacità massima dei cassonetti.
        \item Si assume che tutti i cittadini ricevano la propria smart card per il conferimento di rifiuti.
        \item La messa a terra dei nuovi cassonetti e la distribuzione delle smart card ai cittadini costituiscono un ostacolo temporale.
        \item C'è il rischio che i cittadini non sfruttino i nuovi servizi, ad esempio chiamando il numero verde per prenotare missioni di smaltimento di rifiuti straordinari.
        \item C'è il rischio che i cittadini non sappiano sfruttare la tecnologia delle smart card e non riescano a conferire i propri rifiuti.
    \end{itemize}
}

\subsubsection{Conditions of Satisfaction \& Acceptance Criteria}
Insieme al \textit{Project Overview Statement}, sono state individuate delle condizioni da rispettare per garantire il successo del progetto. In particolare, queste comprendono:
\begin{itemize}
    \item \textbf{Ecosostenibilità}: vista la leadership nei servizi ambientali di \textit{Sphera}, le sue policy interne concedono di realizzare progetti esterni solo con aziende che rispettino determinate caratteristiche di sostenibilità ambientale. Nello specifico, per lo sviluppo del progetto l'azienda deve utilizzare delle \textit{capability} che riducano i consumi energetici almeno del 30\% rispetto ad uno standard di riferimento, che ammettano l'utilizzo di luce naturale per almeno il 50\% delle aree occupate e che abbiano almeno un contratto di forniture fatte con materiali riciclati. Fortunatamente, la nostra azienda soddisfa tali condizioni; sarà quindi solamente necessario mantenerle.
    \item \textbf{Utilizzo di Digital Twin}: il reparto IT del cliente ha concordato con il core team l'utilizzo della tecnologia basata su \textit{digital twin} per garantire compatibilità con estensioni future del progetto o anche funzionalità terze.
    \item \textbf{User Experience}: il committente ha richiesto di coinvolgere degli esperti di design di sua conoscenza per quanto riguarda la resa grafica delle componenti con interfaccia. Ha pertanto chiesto di realizzare dei prototipi (in forma di \textit{mock-up}) da presentare al \textit{design team} da loro scelto per approvazione prima di procedere con la realizzazione dei front end.
\end{itemize}

Inoltre, sono stati stilati dei criteri di accettazione per ogni requisito precedentemente individuato. In particolare:
\begin{itemize}
    \item \textbf{Dumpster Infrastructure}:
        \begin{enumerate}
            \item Tutta l'infrastruttura (cassonetti e punti di raccolta) ha una controparte digitale rappresentata dal proprio \textit{digital twin}.
            \item I cassonetti sono tutti dotati di un sensore di volume, un sensore di peso e un microprocessore connesso alla stazione wi-fi del punto di raccolta, grazie ai quali riescono a comunicare dati ed eventi con le varie componenti del sistema.
            \item I cassonetti si aprono e si chiudono con chiamate da remoto.
        \end{enumerate}
    \item \textbf{Trucks Routing}:
    \begin{enumerate}
        \item Le missioni e i camioncini hanno una controparte digitale rappresentata dal proprio \textit{digital twin}.
        \item Il componente riesce a generare correttamente una missione ordinaria in seguito ad una notifica ricevuta da un cassonetto pieno.
        \item Il componente riesce a generare correttamente missioni straordinarie con frequenza giornaliera.
    \end{enumerate}
    \item \textbf{Admin Dashboard}:
    \begin{enumerate}
        \item \'E presente una pagina in cui si riescono a consultare i punti di raccolta.
        \item \'E presente una pagina in cui si riescono a consultare i camioncini in missione.
        \item \'E presente una pagina in cui si riescono a consultare le richieste di missione straordinaria.
        \item \'E presente una pagina in cui si riescono a consultare i reclami.
    \end{enumerate}
    \item \textbf{Citizen App}:
    \begin{enumerate}
        \item \'E presente una pagina in cui si riescono a consultare i punti di raccolta.
        \item \'E presente una pagina in cui si riescono a prenotare delle richieste di missione straordinaria.
        \item \'E presente una pagina in cui si riescono a consultare le proprie richieste di missione straordinaria.
    \end{enumerate}
\end{itemize}