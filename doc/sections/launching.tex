\section{Launching \& Executing}
\label{sec:launching}

In questa sezione vengono riportati gli elementi significativi del \textit{launch} del progetto.

\subsection{Definizione del Team e delle Regole Operative}

Il team di progetto è stato idealmente definito già in fase di \textit{scoping}, di conseguenza il processo di recruiting non ha richiesto molto tempo. Quindi, una volta confermato il team che ha partecipato alle fasi precedenti, sono stati definiti gli aspetti operativi che accompagnano le parti successive.

In seguito, è stato tenuto un \textit{Kick-Off Meeting} che ha avuto l'obiettivo di riunire tutti gli \textit{stakeholder} del progetto per ripresentare gli aspetti rilevanti del lavoro da svolgere, sotto la prospettiva della pianificazione schedulata.

\subsubsection{Staffing}
I seguenti membri del team, hanno partecipato a tutte le riunioni precedenti e sono, di conseguenza, allineati sulla pianificazione del progetto:
\begin{itemize}
    \item Martina Baiardi (\textit{Converging})
    \item Alessandro Marcantoni (\textit{Accomodating})
    \item Simone Romagnoli (\textit{Diverging})
    \item Marta Spadoni (\textit{Assimilating})
\end{itemize}
Vista l'approvazione della pianificazione da parte del cliente, non è stato ritenuto necessario espandere ulteriormente il \textit{core team}, che, pertanto, coincide con il \textit{developer team}.

\subsubsection{Resource Allocation}
In primo luogo, Alessandro e Simone hanno svolto da \textit{project manager}, quindi sono stati incaricati di tutti gli aspetti non meramente implementativi.
Per quanto riguarda invece

\subsubsection{Operative Rules}
\begin{itemize}
    \item Problem Solving -
    \item Decision Making -
    \item Conflict Resolution -
    \item Meetings -
    \item Comunicazioni -
\end{itemize}

\subsubsection{Project Impact Statement}


