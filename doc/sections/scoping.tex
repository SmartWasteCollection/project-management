\section{Scoping}
\label{sec:scoping}

Per svolgere la fase di scoping al meglio, si simula che il progetto Smart Waste Collection venga proposto dal cliente \textit{Sphera}.
In questo scenario, noi svolgiamo il ruolo di software house e, nello specifico, ricopriamo la posizione di \textit{project manager} con diverse esperienze pregresse.

Nei seguenti capitoli, verrà descritta la richiesta del progetto da parte del cliente, per poi effettuarne un'analisi rappresentata dal Project Overview Statement; tale documento sarà d'interesse sia per i \textit{senior manager} della nostra azienda, sia per quelli dell'azienda cliente.

\subsection{Proposta del Cliente}
\textit{Sphera} è un'azienda multiutility leader nei servizi ambientali, idrici ed energetici nell'ambito di una regione italiana.
Negli ultimi due anni ha riscontrato lamentele da parte dei clienti in merito al servizio di smaltimento dei rifiuti.
In particolare, i cittadini hanno manifestato malcontento a causa dell'inefficienza del servizio stesso e di periodi di
mancato servizio, in cui non sono stati in grado di conferire i rifiuti.
Questo problema è dato dal fatto che i cassonetti sono talvolta pieni, tuttavia la raccolta programmata è prevista dopo alcuni giorni.
\textit{Sphera} stessa si è inoltre accorta che alcune delle raccolte programmate hanno prelevato un quantitativo di rifiuti non sufficiente a giustificare la mobilitazione di un camion.
Inoltre, un altro problema evidenziato dai cittadini consiste nello smaltimento di rifiuti straordinari (ferro, sterpaglie, ecc.).
Nello specifico, al momento è possibile prenotare una spedizione per la raccolta a casa del cittadino in seguito ad una chiamata al numero verde di \textit{Sphera}: le lamentele riguardano principalmente la disponibilità del servizio di call-center e i lunghi tempi d'attesa per l'effettivo smaltimento.
Si richiede quindi un sistema che permetta il monitoraggio dei bidoni stessi indicando così all'azienda il momento più opportuno per la raccolta.
In particolare, il manager dei servizi ambientali Richie Clato ha proposto di introdurre dei nuovi cassonetti con sensori che permettono di misurare il volume e il peso occupati dai rifiuti.
Inoltre, tali cassonetti possono essere aperti solamente tramite l'utilizzo di una smart card fornita ai cittadini.
Date queste premesse, Richie ha identificato i seguenti sistemi software che compongono la soluzione:
\begin{itemize}
    \item \textbf{Dumpster Infrastructure}: i cassonetti saranno dotati di sensori che ne monitorano vari parametri.
    È pertanto necessaria un'infrastruttura che consenta di visualizzare in tempo reale lo stato dei cassonetti sparsi per il territorio.
    La soluzione deve essere scalabile in quanto deve astrarre dal numero di cassonetti.
    Deve inoltre mantenere lo storico degli accessi dei cittadini per futuri utilizzi, a partire da indagini statistiche fino a includere premi per i cittadini modello.
    \item \textbf{Citizen App}: un'applicazione mobile che permetta al cittadino di consultare lo stato dei cassonetti in ciascuna stazione di raccolta per sapere se è possibile conferire un particolare tipo di rifiuto.
    L'applicazione deve presentare una vista "a mappa della città" per permettere al cittadino di orientarsi.
    Inoltre, sarebbe opportuno se nella \textit{home} il cittadino potesse vedere lo stato di alcune stazioni preferite.
    \item \textbf{Trucks Routing}: un'applicazione connessa con la \textbf{Dumpster Infrastructure} che organizza gli itinerari dei camioncini sulla base dello stato dei cassonetti in una determinata zona.
    Quando un autista parte dalla filiera deve avere l'itinerario pronto e deve raccogliere solamente un tipo di rifiuto.
    \item \textbf{Extraordinary Waste Booking}: un'applicazione o sito che permetta ai cittadini di prenotare lo smaltimento di rifiuti straordinari a casa: il cittadino deve indicare l'indirizzo di casa propria, il tipo di rifiuto straordinario e i giorni di disponibilità per la raccolta.
    \item \textbf{Admin Dashboard}: un'applicazione che permetta di visualizzare lo stato di cassonetti e posizione dei camioncini in tempo reale per gli amministratori.
    Deve inoltre poter visualizzare i dati storici e statistiche relative a conferimenti effettuati da cittadini e prenotazioni per raccolte di rifiuti straordinari.
\end{itemize}

\subsection{Knowledge Crunching}
Per studiare al meglio il dominio in questione è stato adottato un approccio \textit{Domain Driven}: nello specifico è stata condotta un'intervista con il cliente coinvolgendo tutti i membri del team e, da quest'ultima, è stata effettuata un'analisi per produrre un diagramma di \textit{impact mapping} mostrato in figura \ref{fig:impact-mapping}. Tale diagramma è d'ausilio alla definizione del \textbf{goal} e ad una prima bozza della \textit{Requirement Breakdown Structure}.
Si riporta di seguito l'intervista con il committente.

\question{Com'è attualmente gestita la raccolta dei rifiuti?}
\answer{Al momento i camioncini raccolgono periodicamente i rifiuti dalle stazioni di raccolta. Noi abbiamo una schedula che elenca quali tipi di rifiuto dobbiamo raccogliere in un dato giorno. Ogni camioncino parte dalla filiera di smaltimento e svuota i cassonetti di tutti i punti di raccolta nella sua \textit{mission area}.
Noi vorremmo ottimizzare la raccolta per permettere ai cittadini di conferire i rifiuti in qualsiasi momento presso il punto di raccolta più vicino.}

\question{Quanti cassonetti ci sono in ogni punto di raccolta?}
\answer{In base al numero di persone che vivono nella zona del punto di raccolta, possono esserci uno o più cassonetti per ogni tipo di rifiuto comune. Per esempio, un punto di raccolta può essere composto da due cassonetti della plastica, due della carta e uno per ogni altro tipo.}

\question{Quali tipi di cassonetti ci sono in ogni punto di raccolta?}
\answer{I tipi di rifiuto presenti in ogni punto di raccolta sono indifferenziata, organico, plastica/alluminio, carta e vetro; tuttavia, stiamo pensando di introdurre altri tipi di rifiuto.
I cassonetti possono avere caratteristiche fisiche diverse, a seconda dal tipo di rifiuto raccolto. Infatti, ci sono due grandezze possibili per i cassonetti: una grande, adatta per rifiuti che non necessitano di essere raccolti con alta frequenza, e una più piccola, che viene usata per i rifiuti organici, ad esempio.
Al momento, i cassonetti più grandi possono essere aperti utilizzando una leva ad altezza dei piedi, mentre quelli più piccoli si aprono manualmente. Tuttavia, vorremmo che i cittadini fossero in grado di aprire tutti i cassonetti utilizzando una smart card. }

\question{Quali tipi di rifiuto vorreste raccogliere? Come?}
\answer{Vorremmo prelevare i rifiuti comuni direttamente dai punti di raccolta. D'altro canto, vorremmo fornire un servizio di raccolta "a casa" *on-demand* per i rifiuti straordinari.
I rifiuti comuni includono:
\begin{itemize}
    \item carta
    \item plastica/alluminio
    \item vetro
    \item indifferenziata
    \item organico
\end{itemize}
I rifiuti straordinari includono:
\begin{itemize}
    \item ramaglie
    \item olio esausto
    \item ferro
    \item elettronica
    \item vestiti
    \item altro
\end{itemize}
}

\question{Quali sono le principali caratteristiche dei cassonetti?}
\answer{Le principali caratteristiche sono:
\begin{itemize}
    \item grandezza: in termini del volume di rifiuti che può contenere (misura in litri)
    \item colore: indica il tipo di rifiuto raccolto
    \item apertura: può essere tramite una leva ad altezza piedi o manuale
\end{itemize}
}

\question{Quali informazioni vorreste tracciare per quanto riguarda un singolo cassonetto?}
\answer{Per ogni cassonetto vorremmo sapere:
\begin{itemize}
    \item il volume occupato
    \item se è aperto o chiuso
    \item se è funzionante o danneggiato
    \item se deve essere svuotato
\end{itemize}
}

\question{Quali tipi di camioncini dei rifiuti avete? Quanti tipi di rifiuti può prelevare un singolo camioncino?}
\answer{Abbiamo solo un tipo di camioncini. Durante una missione, può raccogliere solo un singolo tipo di rifiuto.
Stiamo anche pianificando di acquisire camioncini specifici per i rifiuti straordinari per soddisfare le richieste dei clienti.}

\question{Quanti camioncini dei rifiuti avete?}
\answer{Ce ne sono circa da 10 a 30 parcheggiati in ogni filiera di smaltimento. L'ammontare preciso dipende dal numero di punti di raccolta nella provincia della filiera.}

\question{Quali informazioni vorreste tracciare per quanto riguarda un singolo camioncino dei rifiuti?}
\answer{Vorremmo tracciare:
\begin{itemize}
    \item il volume totale
    \item il volume attualmente occupato
    \item il tipo di rifiuto che si sta raccogliendo al momento
    \item la sua posizione in tempo reale
\end{itemize}
}

\question{Dove vengono portati i rifiuti dai camioncini?}
\answer{I rifiuti vengono portati nella filiera di smaltimento della provincia.}

\question{Quindi i camioncini raccolgono rifiuti da ogni punto di raccolta della provincia?}
\answer{No. A ogni camioncino viene assegnata una missione e, di conseguenza, una *mission area*. Una *mission area* è composta da un insieme di aree residenziale che sono fisicamente vicine tra loro.}

\question{Come sono distribuiti i punt di raccolta nel territorio?}
\answer{C'è una filiera di smaltimento per ogni provincia. Ogni filiera è responsabile dei rifiuti prelevati dai punti di raccolta della propria provincia. Una provincia è suddivisa in aree residenziali dove è collocato un singolo punto di raccolta. Le aree residenziali sono dimensionate in modo da "coprire" più o meno lo stesso numero di persone.}

\question{Quali tipi di rifiuto vengono gestiti in una filiera di smaltimento?}
\answer{Ogni filiera gestisce tutti i tipi di rifiuti. Ciascun tipo di rifiuto viene smaltito da un'apposita catena di smaltimento all'interno della filiera.}

\question{Come volete gestire i rifiuti straordinari?}
\answer{Al momento, i rifiuti straordinari vengono portati direttamente alle filiere dai cittadini. Tuttavia, vorremmo che i cittadini siano in grado di prenotare un appuntamento per raccogliere i rifiuti straordinari direttamente a casa loro.}
