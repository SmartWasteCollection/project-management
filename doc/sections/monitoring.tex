\section{Monitoring \& Control}
\label{sec:monitoring}

In questa sezione vengono riportate le strategie adottate per monitorare i task definiti in fase di pianificazione (\ref{sec:planning}). Inoltre, viene riportato il processo di accettazione del prodotto da parte del committente.

\subsection{Monitoraggio del Progetto}
Il progetto è stato monitorato da diversi punti di vista. Tale attività è stata ritenuta necessaria per rispettare la pianificazione effettuata ed accorgersi presto di eventuali problematiche. Nello specifico, sono stati utilizzati dei metodi formali per monitorare sia lo stato di avanzamento dei task, sia lo stato dei membri del team in termini di problematiche riscontrate e di produttività.

\subsubsection{Reporting}

\subsubsection{Issues Log}

\subsubsection{Meetings}
Come anticipato nella sezione \ref{sec:planning-operative-rules}, quotidianamente sono stati condotti dei \textbf{Project Status Meeting}, ai quali ha partecipato ogni membro del developer team. In queste riunioni non è stato incluso il committente, tuttavia è stato sempre riportato l'esito con una breve descrizione del meeting (massimo 10 righe). Il programma di queste riunioni prevede:
\begin{itemize}
    \item breve presentazione dello stato di avanzamento dei task svolti dai membri del developer team;
    \item aggiornamento dell'\textit{Issues Log};
    \item aggiornamento (se necessario) della schedula.
\end{itemize}



\subsection{Accettazione del Progetto}
