\section{Scoping}
\label{sec:scoping}

Per svolgere la fase di scoping al meglio, si simula che il progetto Smart Waste Collection venga proposto dal cliente \textit{Sphera}.
In questo scenario, noi svolgiamo il ruolo di software house e, nello specifico, ricopriamo la posizione di \textit{project manager} con diverse esperienze pregresse.

Nei seguenti capitoli, verrà descritta la richiesta del progetto da parte del cliente, per poi effettuarne un'analisi rappresentata dal Project Overview Statement; tale documento sarà d'interesse sia per i \textit{senior manager} della nostra azienda, sia per quelli dell'azienda cliente.

\subsection{Proposta del Cliente}
\textit{Sphera} è un'azienda multiutility leader nei servizi ambientali, idrici ed energetici nell'ambito di una regione italiana.
Negli ultimi due anni ha riscontrato lamentele da parte dei clienti in merito al servizio di smaltimento dei rifiuti.
In particolare, i cittadini hanno manifestato malcontento a causa dell'inefficienza del servizio stesso e di periodi di
mancato servizio, in cui non sono stati in grado di conferire i rifiuti.
Questo problema è dato dal fatto che i cassonetti sono talvolta pieni, tuttavia la raccolta programmata è prevista dopo alcuni giorni.
\textit{Sphera} stessa si è inoltre accorta che alcune delle raccolte programmate hanno prelevato un quantitativo di rifiuti non sufficiente a giustificare la mobilitazione di un camion.
Inoltre, un altro problema evidenziato dai cittadini consiste nello smaltimento di rifiuti straordinari (ferro, sterpaglie, ecc.).
Nello specifico, al momento è possibile prenotare una spedizione per la raccolta a casa del cittadino in seguito ad una chiamata al numero verde di \textit{Sphera}: le lamentele riguardano principalmente la disponibilità del servizio di call-center e i lunghi tempi d'attesa per l'effettivo smaltimento.
Si richiede quindi un sistema che permetta il monitoraggio dei bidoni stessi indicando così all'azienda il momento più opportuno per la raccolta.
In particolare, il manager dei servizi ambientali Richie Clato ha proposto di introdurre dei nuovi cassonetti con sensori che permettono di misurare il volume e il peso occupati dai rifiuti.
Inoltre, tali cassonetti possono essere aperti solamente tramite l'utilizzo di una smart card fornita ai cittadini.
Date queste premesse, Richie ha identificato i seguenti sistemi software che compongono la soluzione:
\begin{itemize}
    \item \textbf{Dumpster Infrastructure}: i cassonetti saranno dotati di sensori che ne monitorano vari parametri.
    È pertanto necessaria un'infrastruttura che consenta di visualizzare in tempo reale lo stato dei cassonetti sparsi per il territorio.
    La soluzione deve essere scalabile in quanto deve astrarre dal numero di cassonetti.
    Deve inoltre mantenere lo storico degli accessi dei cittadini per futuri utilizzi, a partire da indagini statistiche fino a includere premi per i cittadini modello.
    \item \textbf{Citizen App}: un'applicazione mobile che permetta al cittadino di consultare lo stato dei cassonetti in ciascuna stazione di raccolta per sapere se è possibile conferire un particolare tipo di rifiuto.
    L'applicazione deve presentare una vista "a mappa della città" per permettere al cittadino di orientarsi.
    Inoltre, sarebbe opportuno se nella \textit{home} il cittadino potesse vedere lo stato di alcune stazioni preferite.
    \item \textbf{Trucks Routing}: un'applicazione connessa con la \textbf{Dumpster Infrastructure} che organizza gli itinerari dei camioncini sulla base dello stato dei cassonetti in una determinata zona.
    Quando un autista parte dalla filiera deve avere l'itinerario pronto e deve raccogliere solamente un tipo di rifiuto.
    \item \textbf{Extraordinary Waste Booking}: un'applicazione o sito che permetta ai cittadini di prenotare lo smaltimento di rifiuti straordinari a casa: il cittadino deve indicare l'indirizzo di casa propria, il tipo di rifiuto straordinario e i giorni di disponibilità per la raccolta.
    \item \textbf{Admin Dashboard}: un'applicazione che permetta di visualizzare lo stato di cassonetti e posizione dei camioncini in tempo reale per gli amministratori.
    Deve inoltre poter visualizzare i dati storici e statistiche relative a conferimenti effettuati da cittadini e prenotazioni per raccolte di rifiuti straordinari.
\end{itemize}