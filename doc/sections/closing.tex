\section{Closing}
\label{sec:closing}

In questa sezione vengono riportati i processi seguiti per chiudere il progetto. Nello specifico, vengono riportate le procedure di accettazione seguite coinvolgendo il committente e i processi di auto-valutazione e \textit{review} del team eseguiti come ultima fase del progetto.

\subsection{Procedure di Accettazione}
Già in fase di pianificazione (\ref{sec:planning}), sono stati accordati dei processi di accettazione del progetto per evitare fraintendimenti ed evitare problematiche di qualsiasi genere. Tali procedure hanno incluso il collaudo delle componenti del progetto, la verifica della corretta installazione del sistema nella sua interezza e la consegna della documentazione formale relativa all'utilizzo e allo sviluppo del progetto. Tali processi sono stati iniziati con il \textit{launching} del progetto (\ref{sec:launching}) grazie ai meccanismi di \textit{monitoring} (\ref{sec:monitoring}) realizzati; tuttavia, il team ha dedicato ulteriore tempo, coinvolgendo il committente, per effettuare verifiche finali a progetto terminato.

\subsubsection{Collaudo}
Il collaudo del progetto ha richiesto che tutti i test sviluppati funzionassero. Il committente ha manifestato la sua fiducia nei test in quanto lo sviluppo del progetto ha avuto un approccio basato su \textbf{Test Driven Development}: ciò ha permesso di accertare il funzionamento di un componente nell'immediato e di costruire test che verificassero il comportamento desiderato del software. La combinazione di tale tecnica con un flusso ben definito di \textbf{Continuous Integration} ha permesso di collaudare continuamente il sistema. Inoltre, ha permesso di identificare problematiche nell'immediato, in particolare con l'invio di notifiche in caso di errori introdotti da cambiamenti o nuove feature nel sistema.

\subsubsection{Deployment}
Una volta accettati i deliverable, si è proceduto con la loro installazione. Anche tale processo è stato automatizzato già dalle fase di precedenti del progetto, grazie alla definizione di un flusso ben definito di \textbf{Continuous Delivery}.

Per l'installazione è stato adottato un approccio \textbf{Cut-Over}, sostituendo completamente le soluzioni precedenti. La scelta di tale approccio è dovuta alla natura del progetto, che aveva come obiettivo quello di cambiare radicalmente le metodologia di raccolta dei rifiuti.

\subsubsection{Documentazione}
Il committente ha richiesto una documentazione molto dettagliata affinché sia possibile apportare modifiche in momenti futuri. Di conseguenza, è stata consegnata una copia del \textbf{Project Notebook}, contenente le descrizioni di tutte le fasi del progetto e una copia di tutti i documenti prodotti (\textit{POS}, \textit{RBS}, etc). Inoltre, sono state consegnate anche le documentazioni più tecniche, come quelle del codice e di come poter utilizzare le componenti sviluppate.

\subsection{Review}
In conclusione, il team si è riunito per effettuare una revisione del lavoro svolto e auto-valutarsi in vista di progetti futuri.

\subsubsection{Post-Implementation Audit}
\'E stata condotta una riunione in cui il team ha valutato quanto il piano realizzato in fase di pianificazione è stato effettivamente rispettato.
Nello specifico, gli obiettivi sono stati raggiunti e il cliente si è ritenuto soddisfatto sia del sistema stesso che dei tempi di consegna, che hanno rispecchiato quelli prestabiliti.

Inoltre, è stata commentata la metodologia di gestione del progetto. L'approccio adottato ha permesso di stabilire il funzionamento del sistema nella sua interezza già nelle fasi iniziali del progetto, per poi concentrarsi sull'implementazione delle componenti in maniera iterativa. Tutti i membri del team hanno concordato il fatto che la metodologia incrementale fosse la più adatta vista la natura del progetto; tuttavia, il gruppo si è sentito più a suo agio in altri progetti con cicli di vita più agili.

\subsubsection{Final Project Report}
Durante la riunione finale è stato redatto un documento riassuntivo che contiene gli elementi significativi di project management adottati. In particolare, sono stati annotati i processi seguiti in ogni \textit{process group} insieme alle considerazioni fatte durante il meeting stesso. Inoltre, a tale report sono stati allegati in appendice tutti i documenti significativi per i project manager.