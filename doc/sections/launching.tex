\section{Launching \& Executing}
\label{sec:launching}

In questa sezione vengono riportati gli elementi significativi riguardanti il \textit{launching} del progetto.

\subsection{Definizione del Team e delle Regole Operative}

Il team di progetto è stato idealmente definito già in fase di \textit{scoping}, di conseguenza il processo di recruiting non ha richiesto molto tempo. Quindi, una volta confermato il team che ha partecipato alle fasi precedenti, sono stati definiti gli aspetti operativi che accompagnano le parti successive.

In seguito, è stato tenuto un \textit{Kick-Off Meeting} che ha avuto come obiettivo quello di riunire tutti gli \textit{stakeholder} del progetto per rimarcare gli aspetti rilevanti del lavoro da svolgere a valle della pianificazione schedulata.

\subsubsection{Staffing}
I seguenti membri del team, hanno partecipato a tutte le riunioni precedenti e sono, di conseguenza, allineati sulla pianificazione del progetto:
\begin{itemize}
    \item Martina Baiardi (\textit{Converging})
    \item Alessandro Marcantoni (\textit{Accomodating})
    \item Simone Romagnoli (\textit{Diverging})
    \item Marta Spadoni (\textit{Assimilating})
\end{itemize}
Vista l'approvazione della pianificazione da parte del cliente, non è stato ritenuto necessario espandere ulteriormente il \textit{core team}, che, pertanto, coincide con il \textit{developer team}.

\subsubsection{Responsibility Management}
Vista la natura orizzontale del team, non è stato ritenuto necessario distinguere le responsabilità dei singoli individui relative alle attività. Tuttavia, sono stati assegnati dei \textbf{ruoli} a ciascun membro del team che permettano di identificare tempestivamente gli \textit{owner} di eventuali problematiche.

In primo luogo, \textit{Alessandro} e \textit{Simone} hanno svolto da \textit{project manager}, quindi sono stati incaricati di tutti gli aspetti non meramente implementativi.
Per quanto riguarda invece gli aspetti più implementativi, \textit{Martina} è stata eletta responsabile delle scelte tecnologiche, che verranno tuttavia confermate dai project manager e discusse assieme al developer team durante i meetings. Invece, \textit{Marta} è stata eletta responsabile del prodotto: si è quindi occupata di gestire i repository di codice prodotti e ha definito come assicurarsi un'alta qualità del codice attraverso processi di \textit{continuous integration} e \textit{continuous delivery}.

\subsubsection{Operative Rules}
\label{sec:planning-operative-rules}
Sono state stabilite delle regole operative per allineare i vari membri del team sulle modalità d'esecuzione dei processi.
\begin{itemize}
    \item \textbf{Meetings} - Ogni mattina verranno condotti degli \textbf{stand-up meeting} che avranno lo scopo di condividere lo stato di avanzamento della realizzazione di task e l'insorgere di eventuali problemi. Inoltre, verranno tenute delle riunioni prima di iniziare la realizzazione di un nuovo \textit{requirement} e al completamento di quest'ultimo; a questi meeting parteciperà anche il committente. Ovviamente, se ritenuto opportuno, sarà possibile richiedere dei meeting da qualsiasi membro e in qualsiasi momento.
    \item \textbf{Problem Solving} - Per risolvere eventuali problemi ci si affiderà a un approccio \textit{team-oriented}. Innanzitutto, verranno discussi durante i meeting giornalieri e si cercherà insieme di individuare le cause e i ``proprietari" dei problemi. Una volta analizzato al meglio il problema, verranno generate idee risolutive con dei \textit{brainstorming} e si svilupperanno dei piani d'azione sulla base di quelle migliori.
    \item \textbf{Decision Making} - Il processo di \textit{decision making} avrà un approccio di tipo \textbf{consultativo}: i project manager saranno ritenuti responsabili delle decisioni prese e delle eventuali conseguenze, tuttavia queste verranno prese solamente dopo essere state discusse con il team.
    \item \textbf{Comunicazioni} - Sono stati concordati diversi canali di comunicazione sulla base degli argomenti di discussione:
    \begin{itemize}
        \item \textit{Whatsapp} per le comunicazioni più veloci.
        \item \textit{Email} per le comunicazioni più formali o con il cliente.
        \item \textit{Slack} per le comunicazioni più tecniche.
    \end{itemize}
    Il cliente ha espressamente dichiarato la sua preferenza per le chiamate (telefoniche o in videoconferenza) in caso di comunicazioni sincrone che lo coinvolgono. Tuttavia, è stato comunque integrato nei canali più interni affinchè sia reso partecipe anche delle notizie minori relative al progetto.
\end{itemize}

\subsubsection{Project Impact Statement}
\label{sec:impact}
I project manager hanno stilato, assieme a tutti i membri del team, un documento che definisca la gestione dei cambiamenti di scope. In particolare, per ogni cambiamento introdotto nelle fasi successive al lancio del progetto, sarà necessario definire:
\begin{itemize}
    \item importanza e pertinenza, stabilendo se il cambiamento conduce a maggiori benefici e se è in linea con gli obiettivi di progetto. È pertanto necessario consultare il cliente prima di prendere una decisione;
    \item costi introdotti, in termini di tempo, risorse umane e denaro;
    \item ridefinizione provvisoria della schedula.
\end{itemize}
I project manager non hanno ritenuto necessario concentrarsi sull'individuazione delle cause che hanno introdotto il cambiamento, data la natura del progetto esterno.

Dopo aver analizzato al meglio il cambiamento, sarà possibile dedurre quanto tempo quest'ultimo detrae dalla \textbf{Scope Bank} allocata. Se il tempo richiesto non è particolarmente eccessivo (non supera il 50\% del valore iniziale della Scope Bank), si potrà procedere seguendo il piano già stilato ed approvato dal cliente.
Nel caso il tempo richiesto sia invece considerevolmente elevato, sarà necessario stabilire insieme al cliente una nuova schedula. Una volta ottenuta l'approvazione finale, verrà formalizzata la nuova schedula e si potrà procedere con l'integrazione del cambiamento.